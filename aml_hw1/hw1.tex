\documentclass[]{article}

\usepackage[utf8]{inputenc}

\usepackage[T1]{fontenc}

\usepackage{geometry}
\geometry{a4paper}

\usepackage{float}
\restylefloat{table}

\usepackage[english]{babel}

\usepackage{amsfonts}
\usepackage{amsmath}

\usepackage{bm}

\usepackage{tikz}
\usetikzlibrary{trees}

\usepackage{algorithm}
\usepackage{algorithmic}

%opening
\title{}
\author{}

\begin{document}

\maketitle

\section{}

We first apply corollary 1 to find:
\begin{equation}
	R(f_{VRM}) \le \hat{R}_{S,\rho}\left(f_{VRM}\right) + \frac{4}{\rho} \sum_{j=1}^{N} \alpha_j \mathfrak{R}_m \left( H_{k \left( j \right)} \right) + \frac{2}{\rho} \sqrt{\frac{\log p}{m}} \left[1+\sqrt{\lceil \log \left[ \frac{\rho^2 m}{\log p} \right]\rceil}\right] + \sqrt{\frac{\log \frac{2}{\delta}}{2m}}
\end{equation}
We then apply corollary 2 and simplify:
\begin{equation}
	R(f_{VRM}) \le R_\rho\left(f_{VRM}\right) + \frac{8}{\rho} \sum_{j=1}^{N} \alpha_j \mathfrak{R}_m \left( H_{k \left( j \right)} \right) + \frac{4}{\rho} \sqrt{\frac{\log p}{m}} \left[1+\sqrt{\lceil \log \left[ \frac{\rho^2 m}{\log p} \right]\rceil}\right] + \sqrt{\frac{2\log \frac{2}{\delta}}{m}}
\end{equation}

\noindent Recall that $f_{VRM} = \underset{f \in \mathcal{F}}{\text{argmin }} \hat{R}_{S,\rho}\left(f_{VRM}\right) + \frac{4}{\rho} \sum_{j=1}^{N} \alpha_j \mathfrak{R}_m \left( H_{k \left( j \right)} \right)$ and so the bound in question:

\begin{equation}
	R(f_{VRM}) \le \underset{f \in \mathcal{F}}{\text{inf}}\left(R_\rho\left(f\right) + \frac{8}{\rho} \sum_{j=1}^{N} \alpha_j \mathfrak{R}_m \left( H_{k \left( j \right)} \right) + \frac{4}{\rho} \sqrt{\frac{\log p}{m}} \left[1+\sqrt{\lceil \log \left[ \frac{\rho^2 m}{\log p} \right]\rceil}\right] + \sqrt{\frac{2\log \frac{2}{\delta}}{m}}\right)
\end{equation}

\noindent holds because of the definition of $f_{VRM}$. For another argument, suppose we have some infimum $f^*$. We could then apply corollary 2 to find:
\begin{equation}
\begin{split}
	\hat{R}_{S,\rho}\left(f^*\right) + \frac{4}{\rho} \sum_{j=1}^{N} \alpha_j \mathfrak{R}_m \left( H_{k \left( j \right)} \right) + \frac{2}{\rho} \sqrt{\frac{\log p}{m}} \left[1+\sqrt{\lceil \log \left[ \frac{\rho^2 m}{\log p} \right]\rceil}\right] + \sqrt{\frac{\log \frac{2}{\delta}}{2m}} \\
	\le R_{\rho}\left(f^*\right) + \frac{8}{\rho} \sum_{j=1}^{N} \alpha_j \mathfrak{R}_m \left( H_{k \left( j \right)} \right) + \frac{4}{\rho} \sqrt{\frac{\log p}{m}} \left[1+\sqrt{\lceil \log \left[ \frac{\rho^2 m}{\log p} \right]\rceil}\right] + \sqrt{\frac{2\log \frac{2}{\delta}}{m}}
\end{split}
\end{equation}

\noindent Once again recall the definition of $f_{VRM}$ and so we know:

\begin{equation}
\begin{split}
\hat{R}_{S,\rho}\left(f_{VRM}\right) + \frac{4}{\rho} \sum_{j=1}^{N} \alpha_j \mathfrak{R}_m \left( H_{k \left( j \right)} \right) + \frac{2}{\rho} \sqrt{\frac{\log p}{m}} \left[1+\sqrt{\lceil \log \left[ \frac{\rho^2 m}{\log p} \right]\rceil}\right] + \sqrt{\frac{\log \frac{2}{\delta}}{2m}} \\
\le \hat{R}_{S,\rho}\left(f^*\right) + \frac{4}{\rho} \sum_{j=1}^{N} \alpha_j \mathfrak{R}_m \left( H_{k \left( j \right)} \right) + \frac{2}{\rho} \sqrt{\frac{\log p}{m}} \left[1+\sqrt{\lceil \log \left[ \frac{\rho^2 m}{\log p} \right]\rceil}\right] + \sqrt{\frac{\log \frac{2}{\delta}}{2m}}
\end{split}
\end{equation}

\noindent Finally we can apply corollary 1 to conclude that the bound in question holds.

\section{}
\section{}
\begin{table}[H]
	\centering
	\caption{Payoff Matrix}
	\label{fig:C3}
	\begin{tabular}{lll}
		& A                        & B                        \\ \cline{2-3} 
		\multicolumn{1}{l|}{A} & \multicolumn{1}{l|}{8,8} & \multicolumn{1}{l|}{9,1} \\ \cline{2-3} 
		\multicolumn{1}{l|}{B} & \multicolumn{1}{l|}{9,1} & \multicolumn{1}{l|}{0,0} \\ \cline{2-3} 
	\end{tabular}
\end{table}
\subsection{}
The pure Nash Equilibria are $\left( A, B \right)$ and $\left( B, A \right)$.
\subsection{}
A mixed Nash Equilibrium is the pair of strategies
\[\left(
    \begin{bmatrix}
      0.5 \\ 0.5 
    \end{bmatrix}
,    \begin{bmatrix}
      0.5 \\ 0.5 
    \end{bmatrix}\right) \]
Because of Symmetricity it is enough to show that the row player has no incentive of deviating from the given strategy. We find the distribution $p =     \begin{bmatrix}
      \rho \\ 1-\rho  
    \end{bmatrix} $ 
That maximizes his payoff:
\begin{align*} 
   \frac{8}{2}\rho + \frac{1}{2}\rho + (1-\rho) \frac{9}{2} + (1-\rho) \frac{0}{2} \\
  &= \frac{9}{2}\rho + \frac{9}{2} - \frac{9}{2}\rho \\
  &= \frac{9}{2}     
\end{align*}
As we can see the payoff is independent of the strategy chosen if the opponent uses the uniform strategy. Thus p constitutes a mixed Nash-Equilibrium with payoffs (4.5,4.5)
\subsection{}
As the game is perfectly symmetric, it suffices to show the following for the row player:
\begin{align*}
p(A,A)u_1(A,A) + p(A,B)u_1(A,B) & = \frac{8}{3}+ \frac{1}{3} \\ &= 3\\ &=   p(B,A)u_1(B,A) \\ &= p(B,A)u_1(B,A) + p(B,B)u_1(B,B) 
\end{align*}                                  
The outcome for each player will be
\[p(A,A)u_1(A,A)+p(A,B)u_1(A,B)+p(B,A)u_1(B,A) = \frac{8}{3}+\frac{1}{3}+\frac{8}{3} = 6\]
This is a better payoff than the payoff of 4.5 found in the previous example.

\end{document}

%%% Local Variables:
%%% mode: latex
%%% TeX-master: t
%%% End:
