\documentclass[]{article}

\usepackage[utf8]{inputenc}

\usepackage[T1]{fontenc}

\usepackage{geometry}
\geometry{a4paper}

\usepackage{float}
\restylefloat{table}

\usepackage[english]{babel}

\usepackage{amsfonts}
\usepackage{amsmath}

\usepackage{bm}

\usepackage{tikz}
\usetikzlibrary{trees}

\usepackage{algorithm}
\usepackage{algorithmic}

%opening
\title{}
\author{}

\begin{document}

\maketitle

\section{}

We first apply corollary 1 to find:
\begin{equation}
	R(f_{VRM}) \le \hat{R}_{S,\rho}\left(f_{VRM}\right) + \frac{4}{\rho} \sum_{j=1}^{N} \alpha_j \mathfrak{R}_m \left( H_{k \left( j \right)} \right) + \frac{2}{\rho} \sqrt{\frac{\log p}{m}} \left[1+\sqrt{\lceil \log \left[ \frac{\rho^2 m}{\log p} \right]\rceil}\right] + \sqrt{\frac{\log \frac{2}{\delta}}{2m}}
\notag
\end{equation}
We then apply corollary 2 and simplify:
\begin{equation}
	R(f_{VRM}) \le R_\rho\left(f_{VRM}\right) + \frac{8}{\rho} \sum_{j=1}^{N} \alpha_j \mathfrak{R}_m \left( H_{k \left( j \right)} \right) + \frac{4}{\rho} \sqrt{\frac{\log p}{m}} \left[1+\sqrt{\lceil \log \left[ \frac{\rho^2 m}{\log p} \right]\rceil}\right] + \sqrt{\frac{2\log \frac{2}{\delta}}{m}}
\notag
\end{equation}

\noindent Recall that $f_{VRM} = \underset{f \in \mathcal{F}}{\text{argmin }} \hat{R}_{S,\rho}\left(f_{VRM}\right) + \frac{4}{\rho} \sum_{j=1}^{N} \alpha_j \mathfrak{R}_m \left( H_{k \left( j \right)} \right)$ and so the bound in question:

\begin{equation}
	R(f_{VRM}) \le \underset{f \in \mathcal{F}}{\text{inf}}\left(R_\rho\left(f\right) + \frac{8}{\rho} \sum_{j=1}^{N} \alpha_j \mathfrak{R}_m \left( H_{k \left( j \right)} \right) + \frac{4}{\rho} \sqrt{\frac{\log p}{m}} \left[1+\sqrt{\lceil \log \left[ \frac{\rho^2 m}{\log p} \right]\rceil}\right] + \sqrt{\frac{2\log \frac{2}{\delta}}{m}}\right)
\notag
\end{equation}

\noindent holds because of the definition of $f_{VRM}$. For another argument, suppose we have some infimum $f^*$. We could then apply corollary 2 to find:
\begin{equation}
\begin{split}
	\hat{R}_{S,\rho}\left(f^*\right) + \frac{4}{\rho} \sum_{j=1}^{N} \alpha_j \mathfrak{R}_m \left( H_{k \left( j \right)} \right) + \frac{2}{\rho} \sqrt{\frac{\log p}{m}} \left[1+\sqrt{\lceil \log \left[ \frac{\rho^2 m}{\log p} \right]\rceil}\right] + \sqrt{\frac{\log \frac{2}{\delta}}{2m}} \\
	\le R_{\rho}\left(f^*\right) + \frac{8}{\rho} \sum_{j=1}^{N} \alpha_j \mathfrak{R}_m \left( H_{k \left( j \right)} \right) + \frac{4}{\rho} \sqrt{\frac{\log p}{m}} \left[1+\sqrt{\lceil \log \left[ \frac{\rho^2 m}{\log p} \right]\rceil}\right] + \sqrt{\frac{2\log \frac{2}{\delta}}{m}}
\end{split}
\notag
\end{equation}

\noindent Once again recall the definition of $f_{VRM}$ and so we know:

\begin{equation}
\begin{split}
\hat{R}_{S,\rho}\left(f_{VRM}\right) + \frac{4}{\rho} \sum_{j=1}^{N} \alpha_j \mathfrak{R}_m \left( H_{k \left( j \right)} \right) + \frac{2}{\rho} \sqrt{\frac{\log p}{m}} \left[1+\sqrt{\lceil \log \left[ \frac{\rho^2 m}{\log p} \right]\rceil}\right] + \sqrt{\frac{\log \frac{2}{\delta}}{2m}} \\
\le \hat{R}_{S,\rho}\left(f^*\right) + \frac{4}{\rho} \sum_{j=1}^{N} \alpha_j \mathfrak{R}_m \left( H_{k \left( j \right)} \right) + \frac{2}{\rho} \sqrt{\frac{\log p}{m}} \left[1+\sqrt{\lceil \log \left[ \frac{\rho^2 m}{\log p} \right]\rceil}\right] + \sqrt{\frac{\log \frac{2}{\delta}}{2m}}
\end{split}
\notag
\end{equation}

\noindent Finally we can apply corollary 1 to conclude that the bound in question holds.	

\section{}
\subsection{}

\subsection{}
First note that:
\begin{equation}
	\underset{q}{\text{min }}\underset{p}{\text{max }}\frac{1}{T}\sum_{t=1}^{T}\underset{a_1 \sim p, a_2 \sim q}{E}\left[u_1\left(\textbf{a}\right)\right] - \frac{R_T}{T}
	\le
	\underset{p}{\text{max }}\frac{1}{T}\sum_{t=1}^{T}\underset{a_1 \sim p, a_2 \sim q_t}{E}\left[u_1\left(\textbf{a}\right)\right] - \frac{R_T}{T}
\notag
\end{equation}
\noindent and:
\begin{equation}
	\underset{q}{\text{min }}\frac{1}{T}\sum_{t=1}^{T}\underset{a_1 \sim p_t, a_2 \sim q}{E}\left[u_1\left(\textbf{a}\right)\right] + \frac{R_T}{T}
	\le
	\underset{p}{\text{max }}\underset{q}{\text{min }}\frac{1}{T}\sum_{t=1}^{T}\underset{a_1 \sim p, a_2 \sim q}{E}\left[u_1\left(\textbf{a}\right)\right] + \frac{R_T}{T}
\notag
\end{equation}

\noindent Now we may apply the inequality from the previous question to see:
\begin{equation}
	\underset{q}{\text{min }}\underset{p}{\text{max }}\frac{1}{T}\sum_{t=1}^{T}\underset{a_1 \sim p, a_2 \sim q}{E}\left[u_1\left(\textbf{a}\right)\right] - \frac{R_T}{T}
	\le
	\underset{p}{\text{max }}\underset{q}{\text{min }}\frac{1}{T}\sum_{t=1}^{T}\underset{a_1 \sim p, a_2 \sim q}{E}\left[u_1\left(\textbf{a}\right)\right] + \frac{R_T}{T}
\notag
\end{equation}

\noindent Now we must consider the bound on $R_T$ for RWM. Happily enough it is $O\left(\sqrt{T \log N}\right)$ and so $\frac{R_T}{T} = O\left(\frac{1}{\sqrt{T}}\right)$ and $\lim\limits_{T \rightarrow \infty} \frac{R_T}{T} = 0$. The other case is symmetric and so this concludes the proof.
\section{}
\begin{table}[H]
	\centering
	\caption{Payoff Matrix}
	\label{fig:C3}
	\begin{tabular}{lll}
		& A                        & B                        \\ \cline{2-3} 
		\multicolumn{1}{l|}{A} & \multicolumn{1}{l|}{8,8} & \multicolumn{1}{l|}{9,1} \\ \cline{2-3} 
		\multicolumn{1}{l|}{B} & \multicolumn{1}{l|}{9,1} & \multicolumn{1}{l|}{0,0} \\ \cline{2-3} 
	\end{tabular}
\end{table}
\subsection{}
The pure Nash Equilibria are $\left( A, B \right)$ and $\left( B, A \right)$.
\subsection{}
A mixed Nash equilibrium is 
\subsection{}

\end{document}
